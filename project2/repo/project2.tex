% 请确保文件编码为utf-8,使用XeLaTex进行编译,或者通过overleaf进行编译

\documentclass[answers]{exam}  % 使用此行带有作答模块
% \documentclass{exam} % 使用此行只显示题目

\usepackage{xeCJK}
\usepackage{zhnumber}
\usepackage{graphicx}
\usepackage{hyperref}
\usepackage{amsmath}
\usepackage{amssymb}
\usepackage{mathtools}
\usepackage{booktabs}
\usepackage{enumerate}
\usepackage{enumitem} % 控制列表样式
\usepackage{listings} 
\usepackage{float}

\title{2025自然语言处理 \\ 课程设计2}
\date{2025.5.1}
\pagestyle{headandfoot}
\author{人工智能学院 221300079 王俊童}
\firstpageheadrule
\firstpageheader{南京大学}{2025自然语言处理}{课程设计2}
\runningheader{南京大学}
{2025自然语言处理}
{课程设计2}
\runningheadrule
\firstpagefooter{}{第\thepage\ 页(共\numpages 页)}{}
\runningfooter{}{第\thepage\ 页(共\numpages 页)}{}

% no box for solutions
% \unframedsolutions
\def \x \mathbf{x}


\setlength\linefillheight{.5in}

% \renewcommand{\solutiontitle}{\noindent\textbf{答:}}
\renewcommand{\solutiontitle}{\noindent\textbf{解:}\par\noindent}

\renewcommand{\thequestion}{\zhnum{question}}
\renewcommand{\questionlabel}{\thequestion .}
\renewcommand{\thepartno}{\arabic{partno}}
\renewcommand{\partlabel}{\thepartno .}

% 实验报告需包含以下内容:

% 实现了哪些方法并对自己设计的代码模块用简洁的语言描述
% 如何复现主要实现结果,包括执行命令和环境依赖,
% 建议修改requirement.txt和新建bash文件来展示如何运行代码
% 不同方法的实验结果如何
% 遇到的具体问题,如何解决
% 对该任务和在探索过程中的思考


\begin{document}
% \normalsize
\maketitle

首先做一个简介:

这次的任务基本是用四个不同方法去尝试把一个解密游戏做好。

\textbf{本来想用免费ai的,比如Free QWQ之类的,但是这一类ai有一个通病就是连接极度不稳定,而且算力分配是有问题的,所以我自费了Kimi,用的模型是中等强度的moonshot-v1-32k。}

\textbf{本次作业制作成本极高。}

同时,由于这个作业评价指标不唯一,所以我设计了三个评价指标
\begin{itemize}
    \item 指标一:是否全部答对。只有0,1二值。
    \item 指标二:在必须回答的问题中,回答对了多少。(非附加问题准确度)
    \item 指标三:在非必须回答问题中,回答对了多少。(附加问题准确度)
\end{itemize}
这个指标设置相对简单,因为我们不去很严格的考虑ai全部能答对,其实这种指标相对的武断了。我们也要考虑到答对了一部分这个情况。

\section{实现方法,对自己设计的代码模块用简洁的语言描述}
\subsection{任务一:调用 API 生成}

首先是调用api生成,这里我重新写了一个框架,基本思路非常非常easy,就是喂进来一个算一个。

当然,ai错的还是有很多的,我们在这里挑选几个case来观察一下准确率:

\textbf{首先是一个完全正确的例子:}

\textbf{然后是一个部分正确的例子:}

\textbf{最后是一个完全错误的例子:}

可以看出ai的推理还是有缺陷的,我们给出在kimi的moonshot-v1-32k下的准确度:

\begin{table}[h]
    \centering
    \begin{tabular}{cccc}
    \toprule
    \textbf{方法} & \textbf{原指标准确度} & \textbf{必答准确度} &\textbf{选答准确度}\\
    \midrule
    API & 2\% & 27.62\% & 27.95\% \\
    \bottomrule
    \end{tabular}
    \caption{方法性能对比一}
\end{table}


\subsection{任务二:Prompt Engineer}

这个任务要求我们化身prompt大师,我们在上面任务的基础上,增加了两个可能帮助我们的大模型进一步生成更准确的推理的prompt:

\textbf{Prompt1: 你是福尔摩斯}:具体操作为:
\begin{lstlisting}[language=python]
    def prompt_sherlock(self, question: str) -> str:
    return f"你是福尔摩斯,接到一个案子:{question} 请详细推理并找出答案。"
\end{lstlisting}

\textbf{Prompt2: 序列化推导问题}:具体操作为:
\begin{lstlisting}[language=python]
    def prompt_step_by_step(self, question: str) -> str:
    return f"请一步步推理以下问题,并给出符合逻辑的正确的最终答案:{question}"
\end{lstlisting}

这两个方法最主要功能就是,提醒大模型你的身份,或者你该怎么做,大模型就不会去乱做或者没有任何先验的情况下乱搞。我们得到的结果如下:

\begin{table}[h]
    \centering
    \begin{tabular}{cccc}
    \toprule
    \textbf{方法} & \textbf{原指标准确度} & \textbf{必答准确度} &\textbf{选答准确度}\\
    \midrule
    Prompt福尔摩斯 & - & - & - \\
    \hline
    Prompt序列推理 & - & - & - \\
    \bottomrule
    \end{tabular}
    \caption{方法性能对比二}
\end{table}


\subsection{任务三:工具使用}

\subsection{任务四:多智能体对话}

编写代码,模拟现实场景中的多智能体交流,让多个 LLM 以不同角色协
同解决问题。这个地方我们主要实现一个法庭的辩论这种情形。

因为这个地方跟破案解密有关,我们设置一个法官和证人这种,然后通过self.loop去控制法官和证人会交锋多少轮次。其实我们想做的就是一个类似于GAN一样的,证人讲证据,然后法官二次判断这样来。然后最后法官作出总结陈词。

首先可以证明的是,因为反复问询这种手段的存在性,我们有必要在不同的loop轮次上进行不同次数的实验,因为这可能意味着也许证人提供更多的细节内容,那么法官会更倾向于做出正确选择。

\textbf{首先从准确率上说}

我们首先看在loop为1的情况下的正确率,这个就是最基本的一种情况了,问一次然后总结。

\begin{table}[h]
    \centering
    \begin{tabular}{cccc}
    \toprule
    \textbf{方法} & \textbf{原指标准确度} & \textbf{必答准确度} &\textbf{选答准确度}\\
    \midrule
    Agent loop1 & - & - & - \\
    \bottomrule
    \end{tabular}
    \caption{方法性能对比四}
\end{table}

那么其实,我们可以增加loop次数,由于经费有限,我们随机挑选5个例子来做这个任务,看看五个例子的预测效果会随着loop变化怎么变。

\begin{table}[h]
    \centering
    \begin{tabular}{cccc}
    \toprule
    \textbf{方法} & \textbf{原指标准确度} & \textbf{必答准确度} &\textbf{选答准确度}\\
    \midrule
    Agent loop1 & - & - & - \\
    \hline
    Agent loop3 & - & - & - \\
    \hline
    Agent loop5 & - & - & - \\
    \bottomrule
    \end{tabular}
    \caption{方法性能对比四}
\end{table}

\textbf{然后从趣味性上说}

我们可以截取一下对话例子:

\section{复现主要实现结果,包括执行命令和环境依赖}

所有的环境依赖都在requirements.txt中.

你可以选择用指令:\textbf{sh run.sh}来运行程序

如果想单独运行:main.py即可。指令如下:

python3 main.py --method 0 -n 0

其中,method0,1,2,3代表4个不同方法,0代表运行所有数据。

\section{不同方法的实验结果如何}

下面给出四个方法的汇总表格实现:
\begin{table}[h]
    \centering
    \begin{tabular}{cccc}
    \toprule
    \textbf{方法} & \textbf{原指标准确度} & \textbf{必答准确度} &\textbf{选答准确度}\\
    \midrule
    API & - & - & - \\
    \hline
    Prompt福尔摩斯 & - & - & - \\
    \hline
    Prompt序列推理 & - & - & - \\
    \hline
    Tools & - & - & - \\
    \hline
    Agent loop1(ALL) & - & - & - \\
    \hline
    Agent loop1(Random 5) & - & - & - \\
    \hline
    Agent loop3(Random 5) & - & - & - \\
    \hline
    Agent loop5(Random 5) & - & - & - \\
    \bottomrule
    \end{tabular}
    \caption{方法性能对比四}
\end{table}


\section{遇到的具体问题,如何解决}

\section{思考}


\end{document}